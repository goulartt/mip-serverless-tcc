% REFERÊNCIAS------------------------------------------------------------------
\usepackage[%
    alf,
    abnt-emphasize=bf,
    bibjustif,
    recuo=0cm,
    abnt-url-package=url,       % Utiliza o pacote url
    abnt-refinfo=yes,           % Utiliza o estilo bibliográfico abnt-refinfo
    abnt-etal-cite=3,
    abnt-etal-list=3,
    abnt-thesis-year=final
]{abntex2cite}                  % Configura as citações bibliográficas conforme a norma ABNT

% PACOTES----------------------------------------------------------------------
\usepackage[utf8]{inputenc}                                 % Codificação do documento
\usepackage[T1]{fontenc}                                    % Seleção de código de fonte
\usepackage{booktabs}                                       % Réguas horizontais em tabelas
\usepackage{color, colortbl}                                % Controle das cores
\usepackage{float}                                          % Necessário para tabelas/figuras em ambiente multi-colunas
\usepackage{graphicx}                                       % Inclusão de gráficos e figuras
\usepackage{icomma}                                         % Uso de vírgulas em expressões matemáticas
\usepackage{indentfirst}                                    % Indenta o primeiro parágrafo de cada seção
\usepackage{microtype}                                      % Melhora a justificação do documento
\usepackage{multirow, array}                                % Permite tabelas com múltiplas linhas e colunas
\usepackage{subeqnarray}                                    % Permite subnumeração de equações
\usepackage{lastpage}                                       % Para encontrar última página do documento
\usepackage{verbatim}                                       % Permite apresentar texto tal como escrito no documento, ainda que sejam comandos Latex
\usepackage{amsfonts, amssymb, amsmath}                     % Fontes e símbolos matemáticos
\usepackage[algoruled, portuguese]{algorithm2e}             % Permite escrever algoritmos em português
%\usepackage[scaled]{helvet}                                % Usa a fonte Helvetica
\usepackage{times}                                          % Usa a fonte Times
%\usepackage{palatino}                                      % Usa a fonte Palatino
%\usepackage{lmodern}                                       % Usa a fonte Latin Modern
\usepackage[bottom]{footmisc}                               % Mantém as notas de rodapé sempre na mesma posição
\usepackage{ae, aecompl}                                    % Fontes de alta qualidade
\usepackage{latexsym}                                       % Símbolos matemáticos
\usepackage{lscape}                                         % Permite páginas em modo "paisagem"
%\usepackage{picinpar}                                      % Dispor imagens em parágrafos
%\usepackage{scalefnt}                                      % Permite redimensionar tamanho da fonte
%\usepackage{subfig}                                        % Posicionamento de figuras
%\usepackage{upgreek}                                       % Fonte letras gregas

% Redefine a fonte para uma fonte similar a Arial (fonte Helvetica)
\renewcommand*\familydefault{\sfdefault}
